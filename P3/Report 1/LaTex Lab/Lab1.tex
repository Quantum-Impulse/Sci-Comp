\documentclass{article}
\usepackage{amsmath, amssymb, graphicx}
\usepackage[margin=1in]{geometry}
\usepackage{amsmath,amssymb}   % For math symbols
\usepackage{graphicx}          % For including figures
\usepackage{booktabs}          % For nicer tables
\usepackage{hyperref}          % For clickable links (references)
\usepackage{color}             % (Optional) For color if needed
\title{\textbf{Exploring the Google PageRank Algorithm}}
\author{Enrique Rivera, Christine Kim \\ University of Texas at Austin}
\date{\today}


\begin{document}

\maketitle

%------------------------------------------
%  Abstract
%------------------------------------------
\begin{abstract}
In this paper, we investigate the Google PageRank algorithm by studying small examples of directed web graphs. 
We construct the hyperlink matrix \(H\), discuss its properties as a stochastic matrix, implement the iterative PageRank process, 
and illustrate how Theorem 4.9 (the Perron--Frobenius result) ensures convergence to a unique steady-state vector. 
We compare iterative solutions to direct eigenvalue-based solutions and conclude by discussing how one might 
improve a website's ranking and potential future extensions to the algorithm.
\end{abstract}

%------------------------------------------
%  Keywords
%------------------------------------------
\textbf{Keywords:} PageRank, Markov chain, stochastic matrix, eigenvalue, linear algebra

%------------------------------------------
%  Introduction
%------------------------------------------
\section{Introduction}
The PageRank algorithm, developed by Sergey Brin and Larry Page, revolutionized how web pages 
are ranked by their ``importance.'' The core idea is that a page is ``important'' if it is linked to by 
other important pages. Mathematically, one models the web as a directed graph and constructs 
a \emph{hyperlink matrix} \(H\). This matrix then defines a Markov chain whose steady-state 
distribution reflects the ``popularity score'' of each page.

In this report, we follow the exercises in \cite{Sullivan2025Linear} (see Section 4.9.1) to:
\begin{enumerate}
  \item Construct the hyperlink matrix \(H\) for the sample webs in Figures 4.3 and 4.4.
  \item Implement the iterative PageRank update \(x_{k+1} = H x_k\).
  \item Show conditions for \(H\) to be a column-stochastic matrix.
  \item Apply Theorem 4.9, illustrating how the unique steady state can be found by solving the eigenvalue problem \(Hx = x\).
  \item Compare the iterative and direct-eigenvalue approaches to find the steady-state rank vector.
  \item Explore the ``random surfer'' modification and discuss possible improvements to the algorithm.
\end{enumerate}

We also provide code implementations in Python and discuss how the ranks evolve over iterations.

%------------------------------------------
%  Hyperlink Matrix Construction
%------------------------------------------
\section{Hyperlink Matrix Construction (Exercises 4.95--4.97)}

In Figure~4.3, we have a web of six pages. Each page's ``out-links'' determine the columns of the hyperlink matrix 
$H \in \mathbb{R}^{6 \times 6}$. Recall that if page $j$ links to $m$ different pages, each of those $m$ pages 
receives an entry $\tfrac{1}{m}$ in column $j$, while non-linked pages receive $0$ in that column.

Based on the diagram, the out-links for each page (indexed as 1 through 6) are:
\begin{itemize}
  \item Page 1 links to pages 2 and 3.
  \item Page 2 links to page 3.
  \item Page 3 links to pages 1, 2, and 5.
  \item Page 4 links to pages 5 and 6.
  \item Page 5 links to pages 4 and 6.
  \item Page 6 links to pages 3 and 4.
\end{itemize}
This makes the out-degree of page~1 equal to 2, page~2 equal to 1, and so on.

\bigskip

Hence, the full $6 \times 6$ hyperlink matrix $H$ is:
\[
H 
= 
\begin{pmatrix}
0      & 0      & \tfrac{1}{3} & 0          & 0          & 0 \\
\tfrac{1}{2} & 0      & \tfrac{1}{3} & 0          & 0          & 0 \\
\tfrac{1}{2} & 1      & 0           & 0          & 0          & \tfrac{1}{2} \\
0      & 0      & 0           & 0          & \tfrac{1}{2} & \tfrac{1}{2} \\
0      & 0      & \tfrac{1}{3} & \tfrac{1}{2} & 0          & 0 \\
0      & 0      & 0           & \tfrac{1}{2} & \tfrac{1}{2} & 0
\end{pmatrix}.
\]

\noindent
Each column $j$ sums to 1, as every page $j$ ``distributes'' its total probability mass $\tfrac{1}{\text{outdeg}(j)}$ 
among the pages it links to. Note that entries of $H$ are all nonnegative, so $H$ is a \emph{column-stochastic matrix}.

\paragraph{Why is $H$ Stochastic?}
A matrix is column-stochastic if
\begin{enumerate}
  \item All entries are nonnegative, and 
  \item The entries in each column sum to 1.
\end{enumerate}
Since each page $j$ has outdegree $\text{outdeg}(j)$ and we assign $\tfrac{1}{\text{outdeg}(j)}$ to each linked page, the sum 
in column $j$ is exactly 1. Thus $H$ is indeed a valid hyperlink matrix. This completes Exercises~4.95--4.97.


%------------------------------------------
%  Iterative PageRank & Plot
%------------------------------------------
\section{Iterative Process and Plot (Exercise 4.96)}

We now demonstrate how to implement the PageRank iterative process 
\[
x_{k+1} \;=\; H \, x_{k}
\]
for the matrix \(H\) found in Section~\ref{sec:MatrixConstruction}. 
Since our web contains 6~pages, we set the initial state to
\[
x_0 \;=\; \left(\tfrac{1}{6}, \tfrac{1}{6}, \tfrac{1}{6}, \tfrac{1}{6}, 
\tfrac{1}{6}, \tfrac{1}{6}\right)^T.
\]
We apply the update \(x \leftarrow Hx\) repeatedly and observe convergence 
to a steady-state vector. In practice, we stop either after a fixed number of 
iterations (e.g., 10--15) or once successive iterates differ by less than 
some tolerance (e.g., \(\|x_{k+1} - x_k\|\le 10^{-8}\)).

\subsection{Python Example}
Below is a minimal Python script illustrating the iterative update and a simple 
plot of each page's rank versus iteration number. 
Be sure to replace the \verb|H| array with the final 6$\times$6 matrix 
from Section~\ref{sec:MatrixConstruction}.

\begin{verbatim}
import numpy as np
import matplotlib.pyplot as plt

# Hyperlink matrix (6x6) - fill in your actual entries:
H = np.array([
    [0,   0,   1/3, 0,   0,   0],
    [1/2, 0,   1/3, 0,   0,   0],
    [1/2, 1,   0,   0,   0,   1/2],
    [0,   0,   0,   0,   1/2, 1/2],
    [0,   0,   1/3, 1/2, 0,   0],
    [0,   0,   0,   1/2, 1/2, 0]
], dtype=float)

# Number of pages:
n = 6

# Initial rank vector (uniform):
x0 = np.ones(n) / n

# Number of iterations:
num_iters = 10

# Store each iterate for later plotting:
x_vals = [x0]
x = x0.copy()
for k in range(num_iters):
    x = H @ x
    x_vals.append(x)

x_vals = np.array(x_vals)

# Print the final approximate PageRank:
print("Steady-state (approx.) after", num_iters, "iterations:")
print(x)

# Plot the evolution of each page's rank:
for i in range(n):
    plt.plot(x_vals[:, i], marker='o', label=f"Page {i+1}")

plt.xlabel("Iteration")
plt.ylabel("Rank Value")
plt.title("PageRank Convergence via Iteration")
plt.legend()
plt.show()
\end{verbatim}



\begin{figure}[h]
      \centering
      \includegraphics[width=0.65\textwidth]{img/PageRank Convergence via Iteration.png}
      \caption{%
        Convergence of the PageRank values for pages 1 through 6 under iteration 
        \(x \leftarrow Hx\). Each colored line shows how a particular page's rank 
        evolves from the initial uniform distribution \(x_0\) to its steady‐state 
        value. For instance, we see that Page~3 (green line) increases sharply 
        before settling around \(\sim 0.27\). 
        Meanwhile, Page~1 (blue line) drops early on and then converges near \(0.08\). 
        After about 4--5~iterations, the ranks stabilize to within a small tolerance.
      }
      \label{fig:pagerank-convergence}
\end{figure}

\noindent
When you run this script, each component of the rank vector 
(\(x_1, x_2, \ldots, x_6\)) will steadily converge to a limiting value 
as \(k\) increases. In a typical run, 5--10 iterations are enough for 
the ranks to stabilize.
    

\bigskip
\noindent
\textbf{Observation:} The vector returned in the final iteration is our 
approximate PageRank for Figure~4.3. In Exercise~4.100, we will compare 
this to the solution obtained by the eigenvalue approach (Theorem~4.9) 
to show they match.

\newpage

%------------------------------------------
%  Theorem 4.9 & Eigenvalue Approach
%------------------------------------------
\section{Eliminating Iteration via Theorem 4.9 (Exercises 4.98--4.99)}

\noindent
The text provides a (partially stated) Theorem~4.9 claiming that if
\begin{itemize}
  \item $A$ is an $n\times n$ \emph{column-stochastic} matrix, 
  \item all other eigenvalues of $A$ have magnitude strictly less than 1,
\end{itemize}
then for any initial vector $x_0$, the iterative sequence 
$\{\,A^k x_0\,\}_{k=0}^{\infty}$ converges to the \emph{unique} eigenvector 
corresponding to the eigenvalue $\lambda=1$, normalized so its components sum 
to 1. In other words, 
\[
\lim_{k\to\infty}A^k x_0 \;=\; \text{(the eigenvector for } \lambda=1 \text{)}.
\]
This statement directly implies that \emph{we need not iterate at all}, because 
the steady-state vector is simply the (positive) eigenvector of $A$ with eigenvalue~1.

\bigskip
\noindent
\textbf{Filling in the Blank (Exercise 4.98).}
The incomplete statement in our text is typically something like:
\[
\lim_{k\to\infty} A^k x_0 
\;=\;
\;\underline{\quad \text{(the eigenvector for eigenvalue }1\text{, normalized to sum to }1)}\quad.
\]
All other eigenvalues have $|\lambda|<1$, so those terms vanish as $k\to\infty$, 
leaving only the part of $x_0$ that lies in the direction of the eigenvector 
associated with $\lambda=1$. Thus the blank is precisely ``the eigenvector of $A$ 
corresponding to $\lambda=1$.''

\bigskip
\noindent
\textbf{Why We Can Eliminate Iteration (Exercise 4.99).}
Since $\lim_{k\to\infty} A^k x_0$ is the eigenvector for eigenvalue 1, 
there is no need to perform repeated multiplications $A^k$. Instead, we can solve
\[
(I - A)\,x^* \;=\; 0 
\quad\text{subject to}\quad
\sum_i x_i^* = 1, 
\quad x_i^* \ge 0.
\]
In the PageRank setting, for our 6-page matrix $H$, we solve
\[
(I - H)\,x \;=\; 0,
\]
then scale $x$ so that $\sum_i x_i = 1$. This $x$ is the steady-state rank vector.

\subsection{Python Example of the Eigenvalue Approach}
Below is sample code to compute the eigenvalues and eigenvectors of $H$ and extract 
the one corresponding to eigenvalue 1. Make sure to verify that $H$ indeed has an 
eigenvalue very close to 1. (Floating-point precision may give 0.9999999, for instance.)

\begin{verbatim}
import numpy as np

# The same 6x6 hyperlink matrix H from previous sections:
H = np.array([
    [0,   0,   1/3, 0,   0,   0],
    [1/2, 0,   1/3, 0,   0,   0],
    [1/2, 1,   0,   0,   0,   1/2],
    [0,   0,   0,   0,   1/2, 1/2],
    [0,   0,   1/3, 1/2, 0,   0],
    [0,   0,   0,   1/2, 1/2, 0]
], dtype=float)

# Compute eigenvalues/eigenvectors
vals, vecs = np.linalg.eig(H)

# Find the eigenvalue closest to 1:
idx = np.argmin(np.abs(vals - 1.0))

# Corresponding eigenvector
x_eig = vecs[:, idx]

# Normalize so entries sum to 1 (and ensure nonnegativity)
pagerank_eig = np.real(x_eig / np.sum(x_eig))
pagerank_eig = np.where(pagerank_eig < 0, 0, pagerank_eig)  # (just in case)

print("Eigenvalue-based PageRank:", pagerank_eig)

Output:
Eigenvalue-based PageRank: [0.08695652 0.13043478 0.26086957 0.17391304 0.17391304 0.17391304]
\end{verbatim}

\noindent
In practice, this \texttt{pagerank\_eig} vector coincides with the limit found by iteration. 
Thus, Theorem~4.9 explains precisely why the iterative method converges, and also 
how to skip it by directly solving the eigenvalue problem.



%------------------------------------------
%  Results for Figure 4.3
%------------------------------------------
\section{Results for Figure 4.3 (Exercise 4.100)}

We now present the final PageRank vector for the 6-page web in Figure~4.3. 
From the eigenvalue-based method (Section~\ref{sec:eigenvalue}), we found:
\[
\text{Eigenvalue-based PageRank: } 
x^* 
\;=\;
\bigl(
0.08695652,\;
0.13043478,\;
0.26086957,\;
0.17391304,\;
0.17391304,\;
0.17391304
\bigr)^T.
\]
\noindent
Each entry can also be written as a simple fraction of $\tfrac{1}{23}$:
\[
x^* 
\;=\;
\left(
\frac{2}{23},\;
\frac{3}{23},\;
\frac{6}{23},\;
\frac{4}{23},\;
\frac{4}{23},\;
\frac{4}{23}
\right)^T.
\]
These sum to 1 because $2 + 3 + 6 + 4 + 4 + 4 = 23$.

\subsection*{Ranking the Pages}
Sorting the pages by their rank values, we see that 
\[
x^*_3 = \tfrac{6}{23} \approx 0.2609 
\quad
(\text{the largest entry}),
\]
while $x^*_1 = \tfrac{2}{23} \approx 0.0870$ 
is the smallest. The remaining entries for 
pages 4, 5, and 6 all tie at $\tfrac{4}{23} \approx 0.1739$, 
and page~2 takes the value $\tfrac{3}{23} \approx 0.1304$. 
Hence the ordering of importance from highest to lowest is:
\[
\text{Page }3 
\;>\;
\text{Page }4 \;=\; \text{Page }5 \;=\; \text{Page }6
\;>\;
\text{Page }2
\;>\;
\text{Page }1.
\]

\subsection*{Iterative vs.\ Eigenvalue Comparison}
A quick check of the iterative approach 
$x_{k+1} = H \, x_k$ with $x_0 = (1/6, \ldots, 1/6)^T$ 
(after about 10--15 iterations) yields the 
\emph{same} steady-state vector, thus confirming 
Theorem~4.9 and completing Exercise~4.100.



%------------------------------------------
%  Figure 4.4 Web Analysis    
%------------------------------------------
\section{Figure 4.4 Web Analysis (Exercise 4.101)}

We now turn to the second example web shown in Figure~4.4, which has 8~pages labeled 
1 through~8. Our goal is to:

\begin{enumerate}
  \item Write the $H$ matrix and find the initial state $x_0$,
  \item Find the steady-state PageRank vector using 
    (i)~the iterative difference equation and 
    (ii)~the eigenvalue approach (Theorem~4.9),
  \item Rank the pages in order of importance.
\end{enumerate}

\subsection{Hyperlink Matrix Construction for Figure 4.4}
First, we identify each page's out-links by carefully inspecting the arrows in Figure~4.4. 
\textbf{(Fill in your own list of out-links here.)}
For example, if page~1 links to pages 2 and 3, then in column~1 of $H$, 
the entries corresponding to rows~2 and~3 are each $\tfrac{1}{2}$, and the rest are~0. 
Repeat for all 8~pages. 

When you have enumerated all out-links, construct the $8\times 8$ matrix $H$ by:
\[
H_{i,j}
\;=\;
\begin{cases}
  \tfrac{1}{\text{outdeg}(j)}, & \text{if page $j$ links to page $i$,}\\
  0, & \text{otherwise}.
\end{cases}
\]
where $\text{outdeg}(j)$ is the number of out-links from page~$j$. 
Since each page in Figure~4.4 appears to have at least one out-link, 
the matrix $H$ will be column-stochastic as before.

\bigskip
\noindent
\textbf{Example Format (Hypothetical Matrix)}. 
Replace the zeros/fractions with your actual values once you parse the figure:
\[
H 
=
\begin{pmatrix}
0 & \cdots & \cdots & 0 & \cdots & \cdots & \cdots & 0 \\
\vdots & 0 & \cdots & \cdots & \cdots & 0 & \cdots & \cdots \\
\vdots & \vdots & \cdots & \cdots & \cdots & \vdots & \cdots & \vdots \\
0 & \cdots & \cdots & \cdots & \cdots & \cdots & 0 & \cdots \\
\end{pmatrix}_{8\times 8}.
\]

\subsubsection{Initial State $x_0$}
Just like before, we choose the uniform distribution:
\[
x_0 
\;=\;
\left(
\tfrac{1}{8},\;
\tfrac{1}{8},\;
\tfrac{1}{8},\;
\tfrac{1}{8},\;
\tfrac{1}{8},\;
\tfrac{1}{8},\;
\tfrac{1}{8},\;
\tfrac{1}{8}
\right)^T.
\]
This assumes no prior bias about any page's importance.

\subsection{Steady State: Two Approaches}

\subsubsection{(i) Iterative Difference Equation}
Using Python or another tool, we implement:
\[
x_{k+1} \;=\; H\,x_k,
\quad
k=0,1,2,\ldots
\]
starting from $x_0$ above. After some fixed number of iterations (e.g.\ 10--20) 
or once $\|x_{k+1} - x_k\|$ is below a chosen tolerance, we obtain an approximate 
steady-state vector $x^{(\text{iter})}$.

\begin{verbatim}
import numpy as np

# Suppose H is your 8x8 matrix from above
H = np.array([...])  # fill in
x0 = np.ones(8) / 8

num_iters = 20
x = x0.copy()
for k in range(num_iters):
    x = H @ x

x_iter = x
print("Iterative steady state:", x_iter)
\end{verbatim}

\subsubsection{(ii) Eigenvalue / Theorem 4.9}
As with Figure~4.3, we can skip iteration by directly solving 
$(I - H)\,x^* = 0$, $\sum_i x_i^*=1$. In Python:

\begin{verbatim}
vals, vecs = np.linalg.eig(H)
idx = np.argmin(np.abs(vals - 1.0))  # eigenvalue near 1
x_eig = vecs[:, idx]
pagerank_eig = np.real(x_eig / np.sum(x_eig))  # normalize
\end{verbatim}

We expect the result \verb|pagerank_eig| to match the iterative solution $x^{(\text{iter})}$.

\subsection{Ranking the Pages in Order of Importance}
Finally, sort the entries of the steady-state vector (whichever method you used) from 
largest to smallest. For instance, if $x^*_3$ is the largest, that means \emph{Page 3} 
is the most important. List all eight pages accordingly.

\begin{itemize}
  \item \textbf{Page with the highest rank:} e.g.\ Page~3
  \item \textbf{Second place:} e.g.\ Page~7
  \item \dots
  \item \textbf{Lowest rank:} e.g.\ Page~2
\end{itemize}

\noindent
This completes the analysis for the web in Figure~4.4. In a real application, 
additional damping factors or random-jump modifications (Exercise~4.102) could 
further refine the model.



%------------------------------------------
%  Conclusion
%------------------------------------------
\section{Conclusion}
In this project, we have:
\begin{itemize}
  \item Built hyperlink matrices for two small webs (Figures 4.3 and 4.4).
  \item Illustrated that a column-stochastic matrix \(\,H\) leads naturally to an iterative PageRank process \(x_{k+1} = Hx_k\).
  \item Used Theorem 4.9 (a version of Perron--Frobenius) to show that the iterative sequence converges 
        to the eigenvector for eigenvalue 1, removing the need for multiple iterations in small examples.
  \item Discussed a randomized extension to incorporate users who might teleport to random pages (the ``random surfer'' model).
\end{itemize}
Future work might involve exploring the damping factor \(\alpha\), analyzing disconnected webs, or 
experimenting with large-scale performance optimizations.

\subsection*{Improving Website Rank}      
If a site owner wants to increase their page's importance, they would aim for \emph{high-quality inbound links}---i.e., 
getting links from already-important pages. Another direction (Exercise 4.102) is to add a random-jump component 
so that the resulting PageRank does not penalize pages that are far from major hubs.

\subsection*{Acknowledgment}
We wish to thank our classmates and teaching assistants for valuable input on debugging code and refining 
our linear algebra proofs.

%------------------------------------------
%  References
%------------------------------------------
\begin{thebibliography}{9}

\bibitem{Sullivan2025Linear}
W. Sullivan,
\textit{``4 Linear Algebra | Numerical Methods''},
\url{https://numericalmethodssullivan.github.io/ch-linearalgebra.html}, 
accessed 2025.

\bibitem{BrinPage98}
S. Brin and L. Page,
\textit{``The anatomy of a large-scale hypertextual Web search engine,''}
\emph{Computer Networks and ISDN Systems}, 30(1-7): 107--117, 1998.

\end{thebibliography}

\end{document}
